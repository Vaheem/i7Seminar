%%%%%%%%%%%%%%%%%%%%%%%%%%%%%%%%%%%%%%%%%%%%%%%%%%%%%%%%%%%%%%%%%%%%%%%%%%%%%%%
% i7 Seminar Report Template
% Version June 23, 2023

\documentclass[a4paper,11pt,DIV=15]{scrartcl} % Do not edit this line.


%%%%%%%%%%%%%%%%%%%%%%%%%%%%%%%%%%%%%%%%%%%%%%%%%%%%%%%%%%%%%%%%%%%%%%%%%%%%%%%
% Preamble

% Page Geometry, Typography and Encoding
\usepackage[utf8]{inputenc}
\usepackage[T1]{fontenc}
\usepackage{microtype}
\renewcommand{\phi}{\varphi}
\renewcommand{\epsilon}{\varepsilon}
% \renewcommand{theta}{\vartheta} % if you want

% Math packages
\usepackage{amsmath}
\usepackage{amssymb}
\usepackage{amsthm}
\usepackage{mathtools}

% Floats
\usepackage{float}
\usepackage{booktabs}
\usepackage{tikz}
\usetikzlibrary{positioning,arrows.meta}

% Colors
\usepackage{xcolor} %already loaded by tikz, but here for completeness
% RWTH colors
% blue violet purple carmine red magenta orange yellow grass cyan gold silver
\definecolor{rwth-blue}{cmyk}{1,.5,0,0}\colorlet{rwth-lblue}{rwth-blue!50}\colorlet{rwth-llblue}{rwth-blue!25}
\definecolor{rwth-violet}{cmyk}{.6,.6,0,0}\colorlet{rwth-lviolet}{rwth-violet!50}\colorlet{rwth-llviolet}{rwth-violet!25}
\definecolor{rwth-purple}{cmyk}{.7,1,.35,.15}\colorlet{rwth-lpurple}{rwth-purple!50}\colorlet{rwth-llpurple}{rwth-purple!25}
\definecolor{rwth-carmine}{cmyk}{.25,1,.7,.2}\colorlet{rwth-lcarmine}{rwth-carmine!50}\colorlet{rwth-llcarmine}{rwth-carmine!25}
\definecolor{rwth-red}{cmyk}{.15,1,1,0}\colorlet{rwth-lred}{rwth-red!50}\colorlet{rwth-llred}{rwth-red!25}
\definecolor{rwth-magenta}{cmyk}{0,1,.25,0}\colorlet{rwth-lmagenta}{rwth-magenta!50}\colorlet{rwth-llmagenta}{rwth-magenta!25}
\definecolor{rwth-orange}{cmyk}{0,.4,1,0}\colorlet{rwth-lorange}{rwth-orange!50}\colorlet{rwth-llorange}{rwth-orange!25}
\definecolor{rwth-yellow}{cmyk}{0,0,1,0}\colorlet{rwth-lyellow}{rwth-yellow!50}\colorlet{rwth-llyellow}{rwth-yellow!25}
\definecolor{rwth-grass}{cmyk}{.35,0,1,0}\colorlet{rwth-lgrass}{rwth-grass!50}\colorlet{rwth-llgrass}{rwth-grass!25}
\definecolor{rwth-green}{cmyk}{.7,0,1,0}\colorlet{rwth-lgreen}{rwth-green!50}\colorlet{rwth-llgreen}{rwth-green!25}
\definecolor{rwth-cyan}{cmyk}{1,0,.4,0}\colorlet{rwth-lcyan}{rwth-cyan!50}\colorlet{rwth-llcyan}{rwth-cyan!25}
\definecolor{rwth-teal}{cmyk}{1,.3,.5,.3}\colorlet{rwth-lteal}{rwth-teal!50}\colorlet{rwth-llteal}{rwth-teal!25}
\definecolor{rwth-gold}{cmyk}{.35,.46,.7,.35}
\definecolor{rwth-silver}{cmyk}{.39,.31,.32,.14}

% Hyperlinks and Cross-References
\usepackage{hyperref}
\usepackage[capitalise,noabbrev]{cleveref}
\hypersetup{%
	pdftoolbar=false,
	pdfmenubar=false,
	colorlinks,
	%pdfborderstyle={/S/U/W 1.25},
	urlcolor={rwth-magenta},
	linkcolor={rwth-red},
	citecolor={rwth-green}
}

\theoremstyle{plain}
\newtheorem{theorem}{Theorem}
\newtheorem{proposition}[theorem]{Proposition}
\newtheorem{lemma}[theorem]{Lemma}
\newtheorem{corollary}[theorem]{Corollary}
\newtheorem{conjecture}[theorem]{Conjecture}
\newtheorem{claim}[theorem]{Claim}
\theoremstyle{definition}
\newtheorem{definition}[theorem]{Definition}
\newtheorem{remark}[theorem]{Remark}



% Misc packages
\usepackage{lipsum}



%%%%%%%%%%%%%%%%%%%%%%%%%%%%%%%%%%%%%%%%%%%%%%%%%%%%%%%%%%%%%%%%%%%%%%%%%%%%%%%
% Document


\begin{document}

%%TODO Insert topic of seminar, e.g. Theoretical Topics in Data Science or Complexity Theory
\subtitle{Seminar ``Theoretical Topics in Data Science''}
\date{\today}
\publishers{RWTH Aachen University}	% Do not edit this line.

%%TODO Change this to your report title.
\title{Latent Semantic Indexing: A Probabilistic Analysis}

%%TODO Change this to your name.
\author{Vahe Eminyan}

\maketitle


%%TODO Provide a short abstract for your report.
\begin{abstract}
This seminar paper gives an analysis of the scientific paper "Latent Semantic Indexing: A Probabilistic Analysis" by Papadimitriou et al.\\
Latent semantic indexing (LSI) is one of the techniques used for information retrieval.
LSI uses a mathematical technique called Singular Value Decomposition (SVD) applied to the term-document matrix. It is used to discover the hidden (latent) structure in the text data.\\
LSI is widely used in practice and has shown strong empirical results. However, due to the large size of term-document matrices, the computation can be slow.
These lead to two interesting questions: why does LSI perform so well, and how can we speed up computations?
Papadimitriou et al. address both of these questions.
They provide a theoretical justification for LSI's effectiveness, considering a special type of term-document matrix.\\
Additionally, to expedite computations, they suggest initially mapping the large matrices into low-dimensional spaces using random projections and then applying LSI to this reduced-dimensional projection.
They prove that the matrix obtained via random projection followed by LSI recovers almost as
much as the matrix obtained by direct LSI, with high probability.\\
In this seminar paper, we will consider all relevant information, providing a detailed analysis of LSI by random projection.





\end{abstract}

\thispagestyle{empty}

\clearpage

%%TODO The content of your report goes below.

\section{Introduction} %KES Ej
In this section, we will introduce the topic and explain how the paper is structured.
My plan is to cover all sections focusing on section 5.


 

\section{LSI Background} %MI EJ
In this part we will explain the technique LSI, how it works, and its mathematical background (SVD).

\section{Probabilistic Corpus Model} %Kes Ej: Since corpus is used for the explanation of both aspects it is more a theoretical framework.
In this section, we will introduce the Probabilistic Corpus Model.

\section{Brief Mention of Analysis of LSI's Good Performance} %KEs Eji see greek paper
In this part, I will provide the most significant information of paragraph 4 of the original paper. \\
Main idea: Under certain conditions and assumptions, it can be shown why LSI performs well.

\section{LSI by Random Projection} %Ereq EJ
In order to reduce the computational time we use dimensionality reduction and then apply LSI. We prove theorem 5 of the paper which maintains that the matrix obtained via random projection followed by LSI recovers almost as
much as the matrix obtained by direct LSI, with high probability.



\section{Conclusion and Summary} %Kes EJic 
In this section, we will draw conclusions based on the findings presented in the seminar paper and give a brief summary of the paper.












\clearpage

\bibliographystyle{plainurl}
\bibliography{references.bib}





\end{document}





